% This is main.tex, a sample paper demonstrating the use of the
% LLNCS macro package for Springer Computer Science proceedings;
% Version 2.20 of 2017/10/04
% 
\documentclass[runningheads]{llncs}
\usepackage[a4paper,left=3cm,right=3cm,top=4cm,bottom=4cm,bindingoffset=5mm]{geometry}
%
% ---- Packages ----
%
\usepackage{graphicx} % enhanced support for graphics
\usepackage{url} % add macros for handling URLs in text
\usepackage[nohyperlinks,nolist]{acronym} % abbreviation utilities
\usepackage{listings}
\usepackage{float}
\usepackage{tabularx}
% TODO: add more packages below if necessary
%
% ---- Acronyms ----
%
\begin{acronym}
\acro{rq}[RQ]{Research Question}
% TODO: define more acronyms here
\acro{llm}[LLM]{large language model}
\acro{lm}[LM]{language model}
\acro{sast}[SAST]{static application security testing}
\acro{ci}[CI]{continuous integration}
\acro{fn}[FN]{false negative}
\acro{fp}[FP]{false positive}
\acro{oss}[OSS]{open-source software}
\acro{rag}[RAG]{retrieval augmented generation}
\acro{cve}[CVE]{Common Vulnerabilities and Exposures}
\acro{bic}[BIC]{bug-inducing commit}
\acro{bfc}[BFC]{bug-fixing commit}
\end{acronym}
%
% ---- Begin Document ----
%
\begin{document}
%
\title{LLM-Augmented Static Analysis Security Testing}
%
%\titlerunning{Abbreviated paper title}
% If the paper title is too long for the running head, you can set
% an abbreviated paper title here
%
% ---- Author Information ----
%
\author{Murilo Escher Pagotto Ronchi}
\institute{Seminar: Software Quality\\
Advisor: Kohei Dozono\\
Technical University of Munich\\
\email{murilo.escher@tum.de}}
%
\maketitle % typeset the header of the contribution
%
% ---- Abstract ----
%
\begin{abstract}
Static Application Security Testing (SAST) tools are widely used to detect vulnerabilities in software systems by analyzing source code, bytecode, or binaries. However, these tools often face limitations, including false positives and difficulties in analyzing vulnerabilities caused by runtime behavior. Recent advancements in large language models (LLMs) have demonstrated their potential to enhance software security analysis, particularly when combined with traditional SAST tools.

This paper explores the integration of SAST tools with LLMs to address these limitations. By leveraging the BugOSS dataset—a collection of open-source projects with documented vulnerabilities—this study employs CodeQL and Infer to analyze vulnerabilities and gathers additional context through call graph analysis. The contextual data, including caller and callee relationships, is combined with SAST output and formatted into structured prompts for GPT-4. This approach aims to reduce false positives and improve the reasoning behind vulnerability detection.

The results reveal that while LLMs excel in reasoning about simpler vulnerabilities like null pointer dereferences, their performance diminishes for more complex issues requiring deeper code comprehension. While contextual information provided valuable insights into the code structure, its overall impact on reducing false positives was limited, as many flagged vulnerabilities were isolated or lacked meaningful interdependencies in the call graph. These findings highlight both the potential and the limitations of LLM-augmented static analysis, laying the groundwork for future research on integrating richer program representations with LLMs for enhanced security testing.

\keywords{Security testing  \and Static analysis \and Large language models.}
\end{abstract}
%
% ---- Text Parts ----
%
\section{Introduction}
\label{sec:intro}
\Acp{llm} have recently gained a lot of popularity, be it in academia, business or for personal use.
Among their many use cases, one currently much researched possibility is their application in \ac{sast}.

\Acl{sast} is a widely used methodology for identifying software vulnerabilities by analyzing the source code, bytecode or binaries of an application.
In contrast to dynamic testing methods, which require the code to be executed, \ac{sast} operates statically.
This characteristic proves useful for integration into the software development life cycle, for example in \ac{ci} pipelines.
However, despite their widespread adoption, \ac{sast} tools face notable limitations.
They are often not capable of identifying all existing vulnerabilities, especially those caused by runtime behavior, like dynamic memory allocation.
Moreover, they usually generate many false positives, which require a lot of effort for the developers to manually go through~\cite{8804441}.

In light of this, the possibility of combining the output from static analyzers with \acp{llm}, as a way to enhance their reasoning and mitigate some of these flaws, is a promising approach being researched.

Recent studies have investigated the use of \acp{llm} for taint analysis~\cite{li2024llmassistedstaticanalysisdetecting}, combined \ac{sast} output with a \ac{rag} system~\cite{du2024vulragenhancingllmbasedvulnerability,keltek2024boostingcybersecurityvulnerabilityscanning}, and compared the effectiveness of both tools~\cite{zhou2024comparisonstaticapplicationsecurity}.
Despite of highlighting their potential in security testing, significant challenges still remain in the way of practical use~\cite{ding2024vulnerabilitydetectioncodelanguage}.
One such challenge is the common necessity of providing context for the vulnerable functions, be it in the form of \ac{rag} systems or the actual relations between the targeted functions and the other ones from the project they are inserted in, as this often proves to be decisive when analyzing software security~\cite{risse2024scorewrongexambenchmarking}.

This paper further investigates the usability of \acp{llm} when combined with \ac{sast} tools and extra context. 
By leveraging a human-curated dataset containing known vulnerabilities in \ac{oss}~\cite{BugOSS}, \ac{sast} tools~\cite{codeql,infer}, and code context, specifically callers and callees of the analyzed functions, acquired through different program representations, this study evaluates the extent to which \acp{llm} can enhance the accuracy and utility of static analysis results.
In this way, it aims to address the challenges presented by false positives in vulnerability detection, contributing to the development of better security systems.
\section{Related Work}
\label{sec:relwork}
This section reviews relevant studies and highlights how this project extends or differs from them, focusing on four key areas: \acl{sast}, the use of \acp{llm} for vulnerability detection, the integration of SAST tools with LLMs, and the importance of providing code context for security analysis.

\textbf{\Acl{sast}}. 
Lipp et al.~\cite{10.1145/3533767.3534380} conducted a comprehensive evaluation of the effectiveness of different \ac{sast} tools, testing their detection capabilities across various vulnerability classes. 
The study involved five static analyzers and proposed different voting systems based on the combination of different tools and examined how they fared in comparison to the use of single analyzers. 
It was concluded that the used analyzers were mostly not capable of detecting real-world vulnerabilities, while combining different analyzers proves very useful for increasing the detection rate, despite also marking more functions as vulnerable. 
Their findings on combining various tools motivated the use of more than one analyzer on this study.

\textbf{Use of \acp{llm} in vulnerability detection}. 
Ding et al.~\cite{ding2024vulnerabilitydetectioncodelanguage} analyzed the usefulness of code \acp{lm} in real-world vulnerability detection. 
They highlighted the ineffectiveness of the current evaluation metrics and created a new dataset, PrimeVul, to combat the limitations they found in existing benchmark datasets. 
Their findings reinforce the belief that code \acp{lm} are ineffective in detecting vulnerabilities and highlight the need for more code context.
However, their study did not investigate how providing additional code might affect detection performance, a gap this study aims to address.

\textbf{Combining \ac{sast} tools with \acp{llm}}.
Li et al.~\cite{li2024llmassistedstaticanalysisdetecting} introduced IRIS, a novel framework based on combining \acp{llm} with static analyzers. The study demonstrated the potential of combining GPT-4 with CodeQL to enhance vulnerability detection. However, their focus was on taint analysis in Java vulnerabilities, whereas this study explores the combination of 2 static analyzers and examines C/C++ code.

\textbf{Providing code context for security testing with \acp{llm}}.
Risse and Böhme~\cite{risse2024scorewrongexambenchmarking} thoroughly analyzed recent top publications in the field of using machine learning for vulnerability detection and argued that their treatment of vulnerability detection as an isolated function-level problem does not reflect real-world vulnerabilities. They highlighted the significance of incorporating calling and code context for effective security analysis. Their results inspired this study to provide the \ac{llm} with extra context in the form of caller and callee relationships for flagged functions.

Keltek et al.~\cite{keltek2024boostingcybersecurityvulnerabilityscanning} combined \acp{llm} with \ac{sast} tools and a knowledge retrieval system based on HackerOne vulnerability reports. They proposed different methods for retrieving similar code and improving their \ac{rag} system. However, their study did not provide the actual code context from the examined functions and relied on synthetic datasets, which may not represent real-world software vulnerabilities.

Du et al.~\cite{du2024vulragenhancingllmbasedvulnerability} created Vul-RAG, a \ac{rag} framework for use in vulnerability detection. They focused on Linux kernel \ac{cve} reports and compared the metrics from different \acp{llm} and the static analyzer Cppcheck. Their study, however, did not include real code context or static analysis results as a knowledge source. Moreover, the benchmark dataset consisted of pairs of functions, where one was vulnerable and the other was a similar, but correct version, which does not correspond to real-world challenges in vulnerability detection.

Sun et al.~\cite{sun2025llm4vulnunifiedevaluationframework} created a \ac{rag} framework based on similar vulnerability reports and the callees for all analyzed functions. Interestingly, they found that additional code context did not necessarily improve the performance results and was even detrimental in some cases. Their study did not examine the effect of integrating static analysis results as contextual information on the \ac{llm} detection capability.
\newpage
\section{Approach}
\label{sec:approach}
This study investigates the potential of combining static analysis tools with \aclp{llm} to enhance the detection of software vulnerabilities while addressing the issue of false positives. 
The methodology involves using two static analyzers, CodeQL and Infer, to flag vulnerabilities, enriching their output with additional code context obtained through a code graph of the project, and leveraging an \ac{llm} to refine the results. 
Figure~\ref{workflow} provides an overview of the workflow.

\begin{figure}[H]
    \centering
    \resizebox{\textwidth}{!}{%
        \includegraphics{figures/workflow.pdf} % Path to your diagram file
    }
    \caption{Overview of the workflow combining SAST tools, contextual analysis, and LLM augmentation.}
    \label{workflow}
\end{figure}

In each of the following subsections, detailed explanations of the tools used will be provided, with the goal of clarifying the diagram illustrated above step by step.

\subsection{Dataset and Ground Truths}
\label{sec:approach:sub:dataset}
BugOSS~\cite{BugOSS} is a dataset containing vulnerabilities manually detected in various \acl{oss} projects.
It comprises a total of 21 different real-world projects, ranging from lesser known projects like Poppler~\cite{poppler}, a library used for rendering PDFs, to widely used software tools, such as Curl~\cite{curl}, a command-line tool that enables data transfer with URLs.

Despite being originally crafted for examinations through fuzz testing, BugOSS contains, for each of the listed projects, information regarding the \ac{bic}, as well as the \ac{bfc}. 
Together with corresponding Dockerfiles and build scripts for each one of the projects, this dataset proves very useful for the reproducibility of the detected vulnerabilities.
Listing~\ref{bugoss} provides a clear overview of how information is structured in the dataset.

\begin{lstlisting}[caption={Hierarchy of the BugOSS dataset}, label={bugoss}]
BugOSS Dataset
+-- Projects (21 Total)
|   \-- Project A (e.g., Poppler)
|   |   +-- Bug-Inducing Commit (BIC)
|   |   +-- Bug-Fixing Commit (BFC)
|   |   +-- Information about Failure Type
|   |   +-- Dockerfile
|   |   +-- Build Script
|   |   \-- Supporting Artifacts
|   \-- Project B (e.g., Curl)
|   |   +-- Bug-Inducing Commit (BIC)
|   |   +-- Bug-Fixing Commit (BFC)
|   |   +-- Information about Failure Type
|   |   +-- Dockerfile
|   |   +-- Build Script
|   |   \-- Supporting Artifacts
\-- Fuzz Testing Metadata
\end{lstlisting}

For this study, the selection and treatment of ground truths followed a systematic process:
\begin{enumerate}
    \item \textbf{Selection Criteria:} Out of the 21 projects in the dataset, only those that could be successfully built and reproduced using the provided Dockerfiles and build scripts were included in the analysis. While the dataset aims to ensure reproducibility, certain projects encountered build errors due to syntax errors in the code or the \ac{bic} not being in a branch. The build process consisted of reverting the repository to the \ac{bic} and using the appropriate build script. Ultimately, 19 projects were successfully built and used as the basis for further analysis.

    \item \textbf{Use of Ground Truths:} The validated vulnerabilities served as the baseline for evaluating the performance of static analysis tools. For each successfully built project, the outputs of the tools were analyzed to determine whether they could detect the known vulnerabilities. The ground truth was considered to be found if the static analyzer flagged a vulnerability in the function containing it, even if not in the same line. This process is discussed in greater detail in the next section.
\end{enumerate}

It is also worthy to mention that the provided Dockerfiles and build scripts needed to be slightly modified. For the Dockerfiles it was necessary to add commands so that both static analyzers were installed during the image creation, and the build scripts needed to be adapted so that the last build command was not directly executed, but instead passed to the corresponding tool, according to their documentation. 

\subsection{Static Analysis with CodeQL and Infer}
\label{sec:approach:sub:sast}
To evaluate the potential of static analysis tools in detecting vulnerabilities, this study utilized two widely known tools: CodeQL and Infer. These tools were selected for their distinct methodologies—CodeQL's query-based analysis and Infer's focus on memory-related issues—and their relevance in the field of software security. Together, they represent complementary approaches to vulnerability detection in real-world software projects.

\subsubsection{CodeQL}
Developed by GitHub, CodeQL is a query-based static analysis tool that generates a database representation of a codebase, enabling developers to query their code as though it were a database. CodeQL excels at identifying vulnerabilities such as:
\begin{itemize}
    \item SQL injection,
    \item Cross-site scripting (XSS), and
    \item Insecure deserialization.
\end{itemize}

For this study, the standard library of 60 predefined queries provided by CodeQL's base installation was used without customization. While highly effective for analyzing web applications and languages like JavaScript and Python, CodeQL's performance in detecting vulnerabilities in C/C++ codebases was limited. This analysis found that CodeQL struggled to detect vulnerabilities relevant to C/C++ projects, such as memory-related issues, and often produced very few or no results for the BugOSS projects.

To run CodeQL, the provided build scripts from the BugOSS dataset were adapted to work with CodeQL's database creation process. Instead of running the \texttt{make} command (or its corresponding equivalent depending on the build setup) directly, the build command was passed to CodeQL's CLI, which intercepted the build process and generated the required database for analysis.

\subsubsection{Infer}
Developed by Meta, Infer is a static analysis tool designed to identify specific classes of vulnerabilities commonly found in C and C++ codebases. Its strengths include detecting:
\begin{itemize}
    \item Null pointer dereferences,
    \item Memory leaks, and
    \item Thread safety violations.
\end{itemize}

Infer uses symbolic execution to analyze paths through the program's control flow, identifying potential bugs. Due to its specialization in memory-related bugs, Infer proved highly effective for analyzing the C/C++ projects in the BugOSS dataset. However, limitations in build system support were observed: the Harfbuzz project could not be analyzed because its Ninja build system is not supported by Infer. This highlights a key challenge when dealing with non-standard or less commonly supported build configurations.

Similar to CodeQL, Infer required modifications to the BugOSS build scripts. Instead of executing the build command directly, the command was passed through Infer's CLI, allowing it to hook into the build process and analyze the code during compilation.

\subsubsection{Comparison and Application in This Study}
Both tools were run on the successfully built projects from the BugOSS dataset. The application process differed:
\begin{itemize}
    \item \textbf{CodeQL}: Required the creation of a database for each project using its build process.
    \item \textbf{Infer}: Analyzed the source code directly without requiring additional preprocessing.
\end{itemize}

The effectiveness of each tool was heavily influenced by the types of vulnerabilities they were designed to detect. Given that the BugOSS dataset consists largely of C/C++ projects:
\begin{itemize}
    \item \textbf{Infer's specialization in memory-related issues} resulted in a significantly higher number of detected vulnerabilities across all projects.
    \item \textbf{CodeQL's default query set} was not well-suited for C/C++ projects and detected very few or no bugs in many cases.
\end{itemize}

This stark difference highlights the importance of tool selection and configuration in vulnerability analysis, especially when analyzing language-specific codebases. The results underline the need to consider the characteristics of the codebase and vulnerabilities when choosing and setting up static analysis tools.

Given that only one ground truth vulnerability was detected across all analyzed projects - specifically, in the PcapPlusPlus project using Infer - this study followed through with PcapPlusPlus for deeper contextual analysis. In addition to the ground truth vulnerability, flagged vulnerabilities detected by Infer in PcapPlusPlus, which were assumed to include false positives, were also analyzed further. The project thus provided a suitable basis for evaluating both the limitations of static analysis tools and the potential to gather contextual information for reducing false positives among flagged vulnerabilities.


\subsection{Contextual Analysis with SVF}
\label{sec:approach:sub:context}
To gather contextual information about the flagged vulnerable functions, the chosen method was to identify which other functions were their callers and callees. This required generating and analyzing the call graph of the code. A call graph is a representation of a program in the form of a control-flow graph. In this structure:
\begin{itemize}
    \item Each function is depicted as a node.
    \item The relationships between functions, such as function calls, are represented as edges connecting the nodes.
\end{itemize}

This representation reveals how vulnerable functions interact with other parts of the code, facilitating the extraction of contextual information.

For this analysis, the focus was on the PcapPlusPlus project, as it was the only project for which a ground truth vulnerability was detected by a static analysis tool (Infer). To generate the call graph, the \textbf{Static Value-Flow Analysis Framework for Source Code (SVF)}~\cite{svf} was used. SVF is a code analysis tool that enables interprocedural dependence analysis for LLVM-based languages. By providing SVF with a bytecode file generated using the \textbf{LLVM compiler}, the framework produces a .dot file that describes the call graph.

\textbf{LLVM} is an open-source compiler framework widely used for program analysis. It compiles source code into an intermediate representation called LLVM bytecode, which can then be analyzed by tools like SVF. The \texttt{.dot} file generated by SVF serves as a textual representation of the call graph. Figure~\ref{callgraph} illustrates how the \texttt{.dot} formatting translates into a visual representation of a call graph, with nodes representing functions and edges representing calls between them.

\begin{figure}[ht]
    \centering
    \begin{minipage}[t]{0.45\textwidth}
        \vspace{0pt} % Ensure no offset at the start of the minipage
        \begin{lstlisting}
digraph G {
    a [label="Function A"];
    b [label="Function B"];
    c [label="Function C"];
    a -> b;
    a -> c;
}
        \end{lstlisting}
        \vspace{0.5em} % Space between lstlisting and custom caption
        {\footnotesize Call graph in .dot file format.} % Manually add caption text
    \end{minipage}
    \hfill
    \begin{minipage}[t]{0.45\textwidth}
        \vspace{0pt} % Ensure no offset at the start of the minipage
        \centering
        \includegraphics[width=\textwidth]{figures/callgraph.pdf}
        \vspace{0.5em} % Optional space for consistency
        {\footnotesize Graphical representation of the call graph.} % Manually add caption text
    \end{minipage}
    \caption{Comparison of the textual and graphical representation of the call graph.}
    \label{callgraph}
\end{figure}

After generating the call graph, the next step was to filter out the vulnerable functions obtained as described in Subsection~\ref{sec:approach:sub:sast}. This involved:
\begin{enumerate}
    \item Using a Python script to match the function names to the corresponding node labels in the graph. This step was necessary because the nodes in the \texttt{.dot} file are not directly named after the functions, as shown in Figure~\ref{callgraph}.
    \item Performing some manual filtering to address mismatches and inconsistencies in the node labeling.
\end{enumerate}

To navigate the graph and extract callers and callees of the vulnerable functions, the \textbf{pydot} library~\cite{pydot} was used. This Python library efficiently handled the parsing of the \texttt{.dot} file and allowed for the lookup of relevant edges and connected functions, making it easier to analyze the relationships between the nodes.

Once the relevant functions and their contexts were extracted, the results were stored in \texttt{.json} files for further analysis. These files contained details about the vulnerable functions and their relationships with other functions in the graph (e.g., callers and callees). 

After that, the source code for all flagged functions and their related functions were was extracted from the repository and written into separate files, so that it could be sent into the LLM for analysis in the next step.

Despite this focus, only 9 out of the 51 vulnerabilities flagged by Infer in PcapPlusPlus appeared in the call graph generated by SVF, and thus were considered for further analysis. This discrepancy highlights a limitation in the representation or the analysis process, as certain vulnerabilities may correspond to code that was not captured or connected in the generated call graph.


\subsection{LLM Augmentation}
\label{sec:approach:sub:llm}
After attaining \ac{sast} results and the contextual information explained in the previous subsections, this data was then put together and formatted into a prompt.

The prompt was formatted as show in Figure 2:

\subsection{Summary}
\label{sec:approach:sub:summary}
This section presented the methodology used to investigate the augmentation of static analysis with large language models for enhanced vulnerability detection. The process was structured into several key steps:
\begin{enumerate}
    \item Dataset Selection: The BugOSS dataset was used as the basis for this study, providing a collection of real-world projects with known vulnerabilities. The dataset enabled reproducible experiments by including bug-inducing and bug-fixing commits alongside build scripts for each project.
    \item Static Analysis: Two widely used static analysis tools, CodeQL and Infer, were employed to detect vulnerabilities. While CodeQL struggled with C/C++ projects, Infer successfully identified a ground truth vulnerability in the PcapPlusPlus project. This project was selected as the focus of the subsequent analysis.
    \item Contextual Analysis: Using SVF, a call graph of the PcapPlusPlus project was generated to extract contextual information about flagged functions. Callers and callees of these functions were identified and consolidated for deeper reasoning about the flagged vulnerabilities.
    \item LLM Augmentation: The outputs from the static analysis tools and contextual analysis were combined into structured prompts for GPT-4o. By leveraging the LLM's reasoning capabilities, this step aimed to reduce false positives and enhance the interpretability of the analysis results.
\end{enumerate}

By integrating static analysis tools, contextual information, and a large language model, this methodology proposes an approach to addressing key limitations of traditional static analysis. The next section evaluates the effectiveness of this approach in terms of its ability to detect vulnerabilities and reduce false positives.
\section{Evaluation}
\label{sec:eval}
Each of the following topics presents the results of the subsections discussed in Section~\ref{sec:approach}.

\subsubsection{Project Builds}
As mentioned in Subsection~\ref{sec:approach:sub:dataset}, only 19 out of the 21 projects contained in BugOSS could be successfully built. These projects were then used with the static analyzers. The fact that not all projects could be built points to a flaw in BugOSS regarding reproducibility.

\subsubsection{Static Analysis Results}
To evaluate the detection capabilities of the static analysis tools, both CodeQL and Infer were run on the 19 successfully built projects. The number of flagged vulnerabilities for each project is summarized in Table~\ref{sast_results}. It is important to denote that Infer could not be executed on Harfbuzz due to its unsupported build system.

\begin{table}[ht]
\centering
\caption{Number of vulnerabilities flagged by CodeQL and Infer for each project.}
\label{sast_results}
\begin{tabular}{|l|c|c|}
\hline
\textbf{Project Name} & \textbf{Flagged by CodeQL} & \textbf{Flagged by Infer} \\
\hline
Arrow & 0 & 71 \\
Aspell & 0 & 63 \\
Curl & 5 & 33 \\
Exiv2 & 3 & 55 \\
File & 0 & 107 \\
Gdal & 152 & 876 \\
Grok & - & - \\
Harfbuzz & 0 & - \\
Leptonica & 1 & 200 \\
Libarchive & 2 & 200 \\
Libhtp & 1 & 15 \\
Libxml2 & - & - \\
Ndpi & 5 & 75 \\
Openh264 & 11 & 191 \\
OpenSSL & 55 & 200 \\
PcapPlusPlus & 1 & 51 \\
Poppler & 15 & 200 \\
Readstat & 3 & 66 \\
Usrsctp & 2 & 0 \\
Yara & 1 & 24 \\
Zstd & 2 & 20 \\
\hline
Total & \textbf{259} & \textbf{2447} \\
\hline
\end{tabular}
\end{table}

As shown in the table, CodeQL flagged significantly fewer vulnerabilities compared to Infer. While Infer specializes in detecting memory-related vulnerabilities, CodeQL's default query set focuses on issues like SQL injection and cross-site scripting, which are uncommon in this dataset. CodeQL's analysis could possibly be improved with customized queries adapted to the nature of the projects in question.

The high amounts of flagged vulnerabilities by Infer is a great example as to why SAST tools present a challenge for a very effective and seamless continuous use. Sifting through so many errors, searching for true positives, proves to be very strenuous and time-consuming for developers. The approach presented in this paper hopes to reduce the number of errors which would then need to be manually checked.

\subsubsection{Gathering Contextual Information}
As previously mentioned 9 of the vulnerabilities flagged by Infer were found in the call graph and used for analysis in this step. Table~\ref{context} provides an overview of how many callers and callees each of the 9 functions had.

\begin{table}[ht]
\centering
\caption{Number of relevant callers and callees found per function analyzed.}
\label{context}
\begin{tabular}{|l|c|c|}
\hline
\textbf{Function} & \textbf{\#Callers} & \textbf{\#Callees} \\
\hline
light\_create\_default\_file\_info & 1 & 0 \\
light\_create\_file\_info & 1 & 0 \\
light\_pcapng\_open\_read & 1 & 0 \\
light\_pcapng\_open\_write & 2 & 0 \\
pcpp::IDnsResource::decodeName & 1 & 4 \\
pcpp::IDnsResource::encodeName & 2 & 4 \\
pcpp::IPFilter::convertToIPAddressWithLen & 1 & 13 \\
pcpp::IPv4Layer::parseNextLayer & 8 & 37 \\
pcpp::IPv6Layer::parseExtensions & 3 & 11 \\
\hline
\end{tabular}
\end{table}

It is important to note that function calls pertaining to the C/C++ standard libraries, the LLVM compiler or any other compilation step, like an address sanitizer, that did not correspond to a function in the PcapPlusPlus repository, were not considered in this analysis, since they went out of the scope of the project.

As explained in Subsections~\ref{sec:approach:sub:context} and~\ref{sec:approach:sub:llm}, the source code for all these functions was gathered in separate files, whose paths were then given in the prompt.

\subsubsection{LLM Analysis}

Table~\ref{fps} shows, for each function, whether GPT-4o flagged the issue as a false positive or not, and the type of vulnerability originally marked by Infer.
\begin{table}[ht]
\centering
\caption{Qualitative Analysis from the LLM.}
\label{fps}
\begin{tabular}{|l|c|c|}
\hline
\textbf{Function} & \textbf{False Positive?} & \textbf{Issue Type} \\
\hline
light\_create\_default\_file\_info & No & NULL\_DEREFERENCE \\
light\_create\_file\_info & No & NULL\_DEREFERENCE \\
light\_pcapng\_open\_read & No & NULL\_DEREFERENCE \\
light\_pcapng\_open\_write & No & NULL\_DEREFERENCE \\
pcpp::IDnsResource::decodeName & Yes & DEAD\_STORE \\
pcpp::IDnsResource::encodeName & Yes & DEAD\_STORE \\
pcpp::IPFilter::convertToIPAddressWithLen & No & DEAD\_STORE \\
pcpp::IPv4Layer::parseNextLayer & No & DEAD\_STORE \\
pcpp::IPv6Layer::parseExtensions & No & NULL\_DEREFERENCE \\
\hline
\end{tabular}
\end{table}

For all \textbf{NULL\_DEREFERENCE} type errors, the LLM provided a similar explanation as to why it is a true positive. For 3 of the 5 cases, a missing Null Check after a memory allocation is missing, and twice there was no check if the return of a function returning a pointer was null or not. All these warning can also be easily verified manually.

Regarding the \textbf{DEAD\_STORE} issues, there are some interesting observations, after human inspection of the issues is done:
\begin{itemize}
    \item For decodeName, GPT-4o outputs, in its conclusion: "The flagged issue appears to be a false positive in terms of security, as decodedNameLength is not used in a way that introduces a vulnerability. However, the presence of the variable as a dead store is indicative of inefficient or unclear code, and removing or repurposing it could improve maintainability.". So, despite detecting an unused value and agreeing with the static analysis result, which is incorrect after manually analyzing the code, the model correctly phrases it as it not actually being a true positive. 
    \item In regards to encodeName, the LLM recognizes the argument flagged by Infer as a pointer which is modified, not needing to be assigned to another variable or returned for it to be considered "used".
    \item Contrary to the previous point, GPT-4o does not recognize the flagged variables in convertToIPAddressWithLen as ones modifying a memory address, which is then later used.
    \item The LLM marks the issues in parseNextLayer as true positives. However, the flagged variables are clearly used, after manual inspection, for choosing which types of layers to instantiate.
\end{itemize}

The observations above highlight the fact that there is still room for improvement regarding the model's reasoning capabilities. The outputs were correct for the most simple type of issue, the null dereference, but not for the ones which demanded greater comprehension of the code.
\section{Conclusion}
\label{sec:conclusion}

This study had the goal of improving static analysis results through the use of contextual information, provided by a call graph, paired with a large language model's reasoning capability. The results point to the fact that more complex issue types, requiring greater understanding of the code, present a real challenge to the model. Moreover, the relevance of the provided context is an important fact to be considered, as it can result in extra clutter for the LLM to digest without improving performance, when not properly chosen.

Future work could investigate the performance increase for CodeQL analysis when using customized queries and examining how other types of context, for example related issues for similar code, impact the LLM's decision-making.
%
% ---- Appendix ----
%
\appendix
% \section{Appendix}
\label{sec:appendix}

Anything additional goes here \dots

Maybe extra information as to how the code words could be provided, if needed.
%
% ---- Bibliography ----
%
\bibliographystyle{splncs04}
\bibliography{library.bib}
%
\end{document}