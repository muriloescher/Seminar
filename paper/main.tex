% This is main.tex, a sample paper demonstrating the use of the
% LLNCS macro package for Springer Computer Science proceedings;
% Version 2.20 of 2017/10/04
% 
\documentclass[runningheads]{llncs}
\usepackage[a4paper,left=3cm,right=3cm,top=4cm,bottom=4cm,bindingoffset=5mm]{geometry}
%
% ---- Packages ----
%
\usepackage{graphicx} % enhanced support for graphics
\usepackage{url} % add macros for handling URLs in text
\usepackage[nohyperlinks,nolist]{acronym} % abbreviation utilities
\usepackage{listings}
% TODO: add more packages below if necessary
%
% ---- Acronyms ----
%
\begin{acronym}
\acro{rq}[RQ]{Research Question}
% TODO: define more acronyms here
\end{acronym}
%
% ---- Begin Document ----
%
\begin{document}
%
\title{Title of Seminar Paper}
%
%\titlerunning{Abbreviated paper title}
% If the paper title is too long for the running head, you can set
% an abbreviated paper title here
%
% ---- Author Information ----
%
\author{Murilo Escher Pagotto Ronchi}
\institute{Seminar: Software Quality\\
Advisor: Kohei Dozono\\
Technical University of Munich\\
\email{murilo.escher@tum.de}}
%
\maketitle % typeset the header of the contribution
%
% ---- Abstract ----
%
\begin{abstract}
The abstract should briefly summarize the contents of the paper in
15--250 words.

\keywords{First keyword  \and Second keyword \and Another keyword.}
\end{abstract}
%
% ---- Text Parts ----
%
\section{Introduction}
\label{sec:intro}
\acp{llm} have recently gained a lot of popularity.
Among their many uses, one currently much researched possibility is their application in \ac{sast}.

Talk about repo-level being more important, \cite{risse2024scorewrongexambenchmarking} 

In this paper, we further investigate their usability when combined with \ac{sast} Tools.
\section{Conclusion}
\label{sec:conclusion}

You can also reference other parts of the document, e.g., sections or subsections.
In Section~\ref{sec:intro} we briefly introduced something, whereas in Subsection~\ref{sec:intro:sub:motivation}, we motivated something else.

Make sure to capitalize chapters, sections or subsections when referencing them.
%
% ---- Appendix ----
%
\appendix
\section{Appendix}
\label{sec:appendix}

Anything additional goes here \dots
%
% ---- Bibliography ----
%
\bibliographystyle{splncs04}
\bibliography{library.bib}
%
\end{document}
