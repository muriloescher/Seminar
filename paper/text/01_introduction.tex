\section{Introduction}
\label{sec:intro}
\Acp{llm} have recently gained a lot of popularity, be it in academia, business or for personal use.
Among their many use cases, one currently much researched possibility is their application in \ac{sast}.

\Acl{sast} is a widely used methodology for identifying software vulnerabilities by analyzing the source code, bytecode or binaries of an application.
In contrast to dynamic testing methods, which require the code to be executed, \ac{sast} operates statically.
This characteristic proves useful for integration into the software development lifecycle, for example in \ac{ci} pipelines.
However, despite their widespread adoption, \ac{sast} tools face notable limitations.
They are often not capable of identifying all existing vulnerabilities, especially those caused by runtime behaviour, like dynamic memory allocation.
Moreover, they usually generate many false positives, which require a lot of effort for the developers to manually go through~\cite{8804441}.

In light of this, the possibility of combining \ac{sast} tools' output with \acp{llm}, as a way to enhance their reasoning and mitigate some of these flaws, is being much researched.

Recent studies have investigated the use of \acp{llm} for taint analysis~\cite{li2024llmassistedstaticanalysisdetecting}, combined \ac{sast} output with a \ac{rag} system~\cite{du2024vulragenhancingllmbasedvulnerability,keltek2024boostingcybersecurityvulnerabilityscanning}, and compared the effectiveness of both tools~\cite{zhou2024comparisonstaticapplicationsecurity}.
Despite of highlighting their potential in security testing, significant challenges still remain in the way of practical use~\cite{ding2024vulnerabilitydetectioncodelanguage}

This paper further investigates the usability of \acp{llm} when combined with \ac{sast} tools. 
By leveraging a human-curated dataset containing known vulnerabilities in \ac{oss}~\cite{BugOSS}, \ac{sast} tools~\cite{codeql,infer}, and code context acquired through different program representations, this study evaluates the extent to which \acp{llm} can enhance the accuracy and utility of static analysis results.
In this way, it aims to address the challenges presented by false positives in vulnerability detection, contributing to the development of better security systems.