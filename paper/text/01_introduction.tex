\section{Introduction}
\label{sec:intro}
\Acp{llm} have recently gained a lot of popularity, be it in academia, business or for personal use.
Among their many use cases, one currently much researched possibility is their application in \ac{sast}.

\Acl{sast} is a widespread methodology for identifying software vulnerabilities by analyzing the source code, bytecode or binaries of an application.
In contrast to dynamic testing methods, which require the code to be executed, \ac{sast} operates statically.
This characteristic proves useful for integration into the software development lifecycle, for example in \ac{ci} pipelines.
\ac{sast} tools, however, are often not capable of identifying all existing vulnerabilities and usually present many \acp{fn} \cite{10.1145/3533767.3534380} as well.

In light of this, the possibility of combining \ac{sast} tools' output with \acp{llm}, as a way to enhance their reasoning and mitigate some of these flaws, is being much researched.
These studies have investigated their use for taint analysis \cite{li2024llmassistedstaticanalysisdetecting}, combined \ac{sast} output with a \ac{rag} system \cite{du2024vulragenhancingllmbasedvulnerability,keltek2024boostingcybersecurityvulnerabilityscanning}, and compared the effectiveness of both tools \cite{zhou2024comparisonstaticapplicationsecurity}.

This paper further investigates their usability when combined with \acp{llm}. 
By leveraging a human-curated dataset containing known vulnerabilities in \ac{oss} \cite{BugOSS}, \ac{sast} tools \cite{codeql,infer}, and code context acquired through different program representations, this study evaluates the extent to which \acp{llm} can enhance the accuracy and utility of static analysis results.